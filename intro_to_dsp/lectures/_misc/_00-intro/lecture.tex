% Copyright 2004 by Till Tantau <tantau@users.sourceforge.net>.
%
% In principle, this file can be redistributed and/or modified under
% the terms of the GNU Public License, version 2.
%
% However, this file is supposed to be a template to be modified
% for your own needs. For this reason, if you use this file as a
% template and not specifically distribute it as part of a another
% package/program, I grant the extra permission to freely copy and
% modify this file as you see fit and even to delete this copyright
% notice. 

\documentclass[aspectratio=169]{beamer}
%\documentclass{beamer}

\setbeamersize{text margin left=10mm, text margin right=2mm}

%\setbeamertemplate{footline}[text line]{%
%  \parbox{\linewidth}{\vspace*{-8pt}some text\hfill\insertshortauthor\hfill\insertpagenumber}}
%\setbeamertemplate{navigation symbols}{}

\defbeamertemplate{headline}{my header}{%
\vskip1pt%
\makebox[0pt][l]{\,\insertshortauthor}%
\hspace*{\fill}\insertshorttitle,/\,\insertshortsubtitle\hspace*{\fill}%
\llap{\insertpagenumber\,/\,\insertpresentationendpage\,}
}
\setbeamertemplate{headline}[my header]


\usepackage{soul}

\setbeamertemplate{itemize items}{-}

%\usepackage{helvet}
\usefonttheme{professionalfonts} % using non standard fonts for beamer
%\usefonttheme{serif} % default family is serif
%\usepackage{fontspec}
%\setmainfont{Liberation Serif}

% There are many different themes available for Beamer. A comprehensive
% list with examples is given here:
% http://deic.uab.es/~iblanes/beamer_gallery/index_by_theme.html
% You can uncomment the themes below if you would like to use a different
% one:
%\usetheme{AnnArbor}
%\usetheme{Antibes}
%\usetheme{Bergen}
%\usetheme{Berkeley}
%\usetheme{Berlin}
%\usetheme{Boadilla}
%\usetheme{boxes}
%\usetheme{CambridgeUS}
%\usetheme{Copenhagen}
%\usetheme{Darmstadt}
%\usetheme{default}
%\usetheme{Frankfurt}
%\usetheme{Goettingen}
%\usetheme{Hannover}
%\usetheme{Ilmenau}
%\usetheme{JuanLesPins}
%\usetheme{Luebeck}
%\usetheme{Madrid}
%\usetheme{Malmoe}
%\usetheme{Marburg}
%\usetheme{Montpellier}
%\usetheme{PaloAlto}
%\usetheme{Pittsburgh}
%\usetheme{Rochester}
%\usetheme{Singapore}
%\usetheme{Szeged}
%\usetheme{Warsaw}



\title{Linear Systems}

% A subtitle is optional and this may be deleted
\subtitle{Introduction}

\author{Sivakumar Balasubramanian}
% - Give the names in the same order as the appear in the paper.
% - Use the \inst{?} command only if the authors have different
%   affiliation.

\institute[Christian Medical College] % (optional, but mostly needed)
{
  \inst{}%
  Department of Bioengineering\\
  Christian Medical College, Bagayam\\
  Vellore 632002
}
% - Use the \inst command only if there are several affiliations.
% - Keep it simple, no one is interested in your street address.

\date{}
% - Either use conference name or its abbreviation.
% - Not really informative to the audience, more for people (including
%   yourself) who are reading the slides online

\subject{Lecture notes on linear systems}
% This is only inserted into the PDF information catalog. Can be left
% out. 

% If you have a file called "university-logo-filename.xxx", where xxx
% is a graphic format that can be processed by latex or pdflatex,
% resp., then you can add a logo as follows:

% \pgfdeclareimage[height=0.5cm]{university-logo}{university-logo-filename}
% \logo{\pgfuseimage{university-logo}}

% Delete this, if you do not want the table of contents to pop up at
% the beginning of each subsection:
\AtBeginSubsection[]
{
  \begin{frame}<beamer>{Outline}
    \tableofcontents[currentsection,currentsubsection]
  \end{frame}
}

% Let's get started
\begin{document}

\begin{frame}
  \titlepage
\end{frame}

% WHAT IS THE COURSE ABOUT?
\begin{frame}{What is the course about?}
\begin{itemize}
\item Introduction to applied linear systems.
\item Introduction to linear systems
\item Focus on state space representation and analysis, state feedback control and state estimation.
\end{itemize}
\end{frame}

% WHAT TO EXPECT FROM THE COURSE?
\begin{frame}{What to expect from the course?}
\begin{itemize}
\item Important concepts in applied linear algebra
\item State space representation and analysis of physical systems
\item Design and analysis of state feedback controllers
\item Design and analysis of linear state observers
\end{itemize}
\end{frame}

% COURSE LAYOUT
\begin{frame}{Course Scoring and Grading}
\begin{columns}[T]
\begin{column}{0.5\textwidth}
\textbf{Course Activities}
\begin{itemize}
\item Homework assignment $15\%$
\item Surprize Quiz $25\%$ 
\item Mid-term $15\%$ 
\item Final $45\%$ 
\end{itemize}
\end{column}

\begin{column}{0.5\textwidth}
\textbf{Grading policy}: \textbf{\ul{No relative grading}}
\begin{itemize}
\item A+: Score $\geq 90/100$
\item A: $80 \leq$ Score $< 90$
\item B: $70 \leq$ Score $< 80$
\item C: $60 \leq$ Score $< 70$
\item D: $50 \leq$ Score $< 60$
\item E: $40 \leq$ Score $< 50$
\item F: Score $< 40$
\end{itemize}
\end{column}
\end{columns}
\end{frame}

% COURSE CONTENT
\begin{frame}{Course content}
\begin{columns}[T]
\begin{column}{0.5\textwidth}
\textbf{Applied Linear Alegbra}
\begin{itemize}
\item Vectors
\item Matrices
\item Least squares methods
\item Eigenvectors and eigenvalues
\item Matrix norm, Positive definiteness
\item Singular Value Decomposition
\end{itemize}
\textbf{State Space Representation and Analysis}
\begin{itemize}
\item Linear dynamical systems (LDS)
\item Modelling physical systems
\item Solution to LDS
\end{itemize}
\end{column}
\begin{column}{0.5\textwidth}  %%<--- here
\begin{itemize}
\item Stability
\item Controllability
\item Observability
\end{itemize}
\textbf{Controller and Observer Design}
\begin{itemize}
\item State feedback control
\item Linear observers
\item \textit{Linear quadratic regulators}
\item \textit{Kalman Filter}
\end{itemize}
\end{column}
\end{columns}
\end{frame}

\end{document}