\documentclass[9pt]{article}

\usepackage[utf8]{inputenc}
\usepackage{geometry}
\geometry{
    a4paper,
    total={170mm,257mm},
    left=15mm,
    right=15mm,
    top=20mm,
    bottom=20mm,
}
\usepackage{multicol}
\usepackage[font=small,labelfont=bf]{caption}
\setlength{\columnsep}{0.25cm}
\usepackage[inline]{enumitem}
\usepackage{amssymb}
\usepackage{xcolor}
\usepackage{mathtools} 
\setlength{\parindent}{0em}
\setlength{\parsep}{0em}
\usepackage{tikz}
\setlength{\parskip}{0em}
\usetikzlibrary{decorations.pathmorphing,patterns}
\usepackage[american,cuteinductors]{circuitikz}
\usetikzlibrary{shapes,arrows,circuits,calc,babel}
% Definition of blocks:
\tikzset{%
  block/.style    = {draw, thick, rectangle, minimum height = 3em,
    minimum width = 3em},
  sum/.style      = {draw, circle, node distance = 2cm}, % Adder
  input/.style    = {coordinate}, % Input
  output/.style   = {coordinate} % Output
}
% Defining string as labels of certain blocks.
\newcommand{\suma}{\Large$+$}
\newcommand{\inte}{$\displaystyle \int$}
\newcommand{\derv}{\huge$\frac{d}{dt}$}

\def\mf{\ensuremath\mathbf}
\def\mb{\ensuremath\mathbb}
\def\mc{\ensuremath\mathcal}
\def\lp{\ensuremath\left(}
\def\rp{\ensuremath\right)}
\def\lv{\ensuremath\left\lvert}
\def\rv{\ensuremath\right\rvert}
\def\lV{\ensuremath\left\lVert}
\def\rV{\ensuremath\right\rVert}
\def\lc{\ensuremath\left\{}
\def\rc{\ensuremath\right\}}
\def\ls{\ensuremath\left[}
\def\rs{\ensuremath\right]}
\def\bmx{\ensuremath\begin{bmatrix*}[r]}
\def\emx{\ensuremath\end{bmatrix*}}
\def\bmxc{\ensuremath\begin{bmatrix*}[c]}
\def\emxc{\ensuremath\end{bmatrix*}}
% \def\t{\lp t\rp}
% \def\k{\ls k\rs}

\newcommand{\demoex}[2]{\onslide<#1->\begin{color}{black!60} #2 \end{color}}
\newcommand{\demoexc}[3]{\onslide<#1->\begin{color}{#2} #3 \end{color}}
\newcommand{\anim}[3]{\onslide<#1->{\begin{color}{#2!60} #3 \end{color}}}
\newcommand{\ct}[1]{\lp #1\rp}
\newcommand{\dt}[1]{\ls #1\rs}

\renewcommand{\familydefault}{\sfdefault}

\begin{document}
\begin{center}
    \begin{Large}
        \textbf{Introduction to DSP: Mathematical Preliminaries - Assignment}
    \end{Large}
\end{center}
\vspace{0.2cm}

\begin{multicols}{2}
    \begin{enumerate}
        \item  Consider a complex number, $z = x + jy$. We know that this complex number can also be written as $z = {re}^{j\theta}$, which is the called the Euler representation of the complex number. Find the Euler representation for the following complex numbers:
        \begin{enumerate}
            \item $z = -1 - 1j$
            \item $z = - 100j$
            \item $z = 10\cos(\frac{\pi}{4}) + j10\sin(\frac{\pi}{4})$
        \end{enumerate}

        Plot these complex number in the complex plane.

        \item The previous two complex numbers are fixed complex numbers. Let us now consider a complex number that changes with time $t \in \mathbb{R}$, and we write this complex numbers as $z(t)$ to indicate that it is a function of time $t$. $z(t)$ can be thought of as a mathematical function, that maps from the set of real number $(\mathbb{R})$ (time) to the set of complex numbers $(\mathbb{C})$. Find the Euler representation of the following time-varying complex numbers. Note that the $r$ and $\theta$ will be time-varying as well.
        \begin{enumerate}
            \item $z(t) = 1 + t + j2t$
            \item $z(t) = \cos(\pi t) + j\sin(\pi t)$
            \item $z(t) = \cos(\pi t) + j2\sin(\pi t)$
            \item $z(t) = t\cos(\pi t) + jt\sin(\pi t)$
        \end{enumerate}

        \item Plot the trajectory of the complex number $z = t\,e^{j2\pi t}$ for time $0 \leq t \leq 5$.

        \item For each complex number $z = x + jy$, we can define a complex conjugate representd by $\overline{z}$, which is defined as $\overline{z} = x -jy$, i.e the sign of the imaginary component is changed, while the real component remains unchanged. For the complex number $z = 3 + 4j$, plot the $z$ and $\overline{z}$ on the complex plane.

        \item If a complex number $z = re^{j\theta}$, what is the correponding Euler representation for its complex conjugate $\overline{z}$.

        \item For any complex number $z = x + jy$, what can you say about the outcomes of the following operations:
        \begin{enumerate}
            \item $z + \overline{z}$
            \item $z - \overline{z}$
            \item $j(z - \overline{z})$
            \item $z\overline{z}$
        \end{enumerate}

        \item Consider two complex numbers $z_1 = 1 - i$ and $z_2 = 2 + i$. Using the complex plane, explain geometrically what happens we add the two complex numbers $\left(z_1 + z_2\right)$, and multipy the two complex numbers $\left(z_1z_2\right)$.

        \item  Write down the domain and range of the following functions:
        \begin{enumerate}
            \item $y = \log\left(x\right)$
            \item $y = \sqrt{x - 5}$
            \item $y = \sin\left(x\right)$
        \end{enumerate}
    \end{enumerate}
    \vfill
\end{multicols}

\end{document}