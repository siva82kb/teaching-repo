\documentclass[9pt]{article}

\usepackage[utf8]{inputenc}
\usepackage{geometry}
\geometry{
    a4paper,
    total={170mm,257mm},
    left=15mm,
    right=15mm,
    top=20mm,
    bottom=20mm,
}
\usepackage{multicol}
\usepackage[font=small,labelfont=bf]{caption}
\setlength{\columnsep}{0.25cm}
\usepackage[inline]{enumitem}
\usepackage{amssymb}
\usepackage{xcolor}
\usepackage{mathtools} 
\setlength{\parindent}{0em}
\setlength{\parsep}{0em}
\usepackage{tikz}
\setlength{\parskip}{0em}
\usetikzlibrary{decorations.pathmorphing,patterns}
\usepackage[american,cuteinductors]{circuitikz}
\usetikzlibrary{shapes,arrows,circuits,calc,babel}
% Definition of blocks:
\tikzset{%
  block/.style    = {draw, thick, rectangle, minimum height = 3em,
    minimum width = 3em},
  sum/.style      = {draw, circle, node distance = 2cm}, % Adder
  input/.style    = {coordinate}, % Input
  output/.style   = {coordinate} % Output
}
% Defining string as labels of certain blocks.
\newcommand{\suma}{\Large$+$}
\newcommand{\inte}{$\displaystyle \int$}
\newcommand{\derv}{\huge$\frac{d}{dt}$}

\def\mf{\ensuremath\mathbf}
\def\mb{\ensuremath\mathbb}
\def\mc{\ensuremath\mathcal}
\def\lp{\ensuremath\left(}
\def\rp{\ensuremath\right)}
\def\lv{\ensuremath\left\lvert}
\def\rv{\ensuremath\right\rvert}
\def\lV{\ensuremath\left\lVert}
\def\rV{\ensuremath\right\rVert}
\def\lc{\ensuremath\left\{}
\def\rc{\ensuremath\right\}}
\def\ls{\ensuremath\left[}
\def\rs{\ensuremath\right]}
\def\bmx{\ensuremath\begin{bmatrix*}[r]}
\def\emx{\ensuremath\end{bmatrix*}}
\def\bmxc{\ensuremath\begin{bmatrix*}[c]}
\def\emxc{\ensuremath\end{bmatrix*}}
% \def\t{\lp t\rp}
% \def\k{\ls k\rs}

\newcommand{\demoex}[2]{\onslide<#1->\begin{color}{black!60} #2 \end{color}}
\newcommand{\demoexc}[3]{\onslide<#1->\begin{color}{#2} #3 \end{color}}
\newcommand{\anim}[3]{\onslide<#1->{\begin{color}{#2!60} #3 \end{color}}}
\newcommand{\ct}[1]{\lp #1\rp}
\newcommand{\dt}[1]{\ls #1\rs}

\renewcommand{\familydefault}{\sfdefault}

\begin{document}
\begin{center}
    \begin{Large}
        \textbf{Introduction to DSP: Systems - Assignment: Digital Filters}
    \end{Large}
\end{center}
\vspace{0.2cm}

\begin{multicols}{2}
    \begin{enumerate}
        \item \textbf{All pass filter}. Show that a filter of the following transfer function has a cosntant magnitude response for all frequencies.
        \[ H\lp z \rp = \frac{z^{-1} + p}{1 - pz^{-1}}, \,\,\, p \in \mb{R} \]

        Write python program to the plot the mangitude and phase response of this filter.
        
        \item Consider a bandlimited the time domain signal $x_c(t)$ that is sampled satisfying the Nyquist criterion. 
        \[  x_d[n] = x_c(n\cdot T_s), \,\,\, T_s \text{ is the sampling time} \]

        Write down the frequency response of a filter $H$ that delays an input signal $x_d[n]$ by 2 samples, such that
        \[ y[n] = H\{ x_d[n] \} = x_d[n - 2] = x_c(n\cdot T_s - 2\cdot T_s) \]

        Find the impulse response of this filter using the inverse DTFT formula.

        It is possible to design a filter $\hat{H}$ that delays the input signal by a fraction of a sample, i.e
        \[ \hat{y}[n] = \hat{H}\{ x_d[n] \} = x_c(n\cdot T_s - 0.5 \cdot T_s) \]

        Write down the frequenct response of such a filter, and derive it impulse response.

        \item \textbf{Notch filter}. A notch filter is filter that selectively remove on more frequency components in the incoming singal. These filters have a magnitude response of $0$ at specific frequencies.

        In a system of interest, continous-time signal are sampled at 500 Hz, and are processed using a computer. The incoming data is corrupted by the 50Hz power line interfernce. We are interest in designing a simple FIR filter for removing the 50Hz interference. We can accomplish this using the following filter,
        \[ H(z) = 1 - 2\cos\lp \Omega \rp z^{-1} + z^{-2} \]

        Where are the poles and zeros of this transfer function? Choose the value of $\Omega$ such that the digital filter $\lvert H\lp \Omega \rp = 0$, such that 50Hz interference in the sampled data is removed.

        Write a python program to plot the magnitude and phase response of this filter as a function of real frequency.

        You will find that this filter introduces signifiant attenuation to the frequencies around 50Hz. One approach to address this issue is to develop a FIR filter using the frequency sampling method, by defining a desired frequency response for a set $N$ discrete frequencies, obtain the impulse response for such a filter. Write a python program to design a FIR notch filter of length $N=100$ to remove the 50Hz powerline interference.

        Plot the magnitude and phase response of this filter as a function of real frequency.

    \end{enumerate}
    \vfill
\end{multicols}

\end{document}